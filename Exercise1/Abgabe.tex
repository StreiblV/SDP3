
\include{Definition} %beinhaltet alle benötigten Packages etc.
\begin{document}

\title{SDP - Exercise 01} % Übungsname und Nummer angeben
\subtitle{winter semester 2019/20} % Semester angeben oder auskommentieren, falls nicht erwünscht
\author{
Viktoria Streibl - S1810306013\\
  Daniel Weyrer - S1820306047
} % Autorenname
\date{\today} % Das heutige Datum automatisch einfügen

\maketitle % Titelseite erstellen

\newpage
\tableofcontents % Inhaltsverzeichnis erstellen
\newpage

\section{Organizational}
\subsection{Team}
\begin{itemize}
	\item Viktoria 	Streibl 		- 	S1810306013
	\item Daniel 	Weyrer		-	S1820306044
\end{itemize}

\subsection{Roles and responsibilities}

\subsubsection{Jointly}
\begin{itemize}
	\item planning
	\item testing (Testdriver)
	\item Documentation
	\item Systemdocumentation
\end{itemize}

\subsubsection{Viktoria Streibl - 10 ph}
\begin{itemize}
	\item Base Class for Vehicles
	\item Derived Classes
		\subitem Class Motorcycle
		\subitem Class Car
		\subitem Class Truck
	\item Class Logbook
		\subitem plausibility test of input data (current Mileage and Date)
\end{itemize}

\subsubsection{Daniel Weyrer - 8 ph}
\begin{itemize}
	\item Main Class Carpool
	\item Class Diagram
\end{itemize}

% Für eine numerierte Aufzählung verwendet man 

%\begin{enumerate}
%	\item 
%\end{enumerate}

\section{Requirenment Definition(System Specification)}

%Hier wird die Aufgabenstellung beschrieben. Sie ist gegebenfalls aus der Systemspezifikation abzuleiten und in kompakter Form zu halten. Man sollte hier in kurzen und
%prägnanten Worten lesen können, worum es überhaupt geht. Neben den Anforderungen und Zielen steht hier alles, was von fundamentaler Bedeutung für das Verständnis
%eines nicht versierten Lesers (der aber sehr wohl über einschlägige Kenntnisse verfügt) nötig ist. Alles andere entfällt hier.

It was a carpool desired the various types of vehicles includes, such as cars, trucks and motorcycles. Each vehicle type should also include and output some key data such as car make, license plate and fuel type. In addition, each vehicle must keep a logbook and enter the kilometers driven in one day.
Any number of vehicles can be added and deleted in the program. You can search for the license plate and output all vehicles with the basic data.

\section{System Design}
%Hier erfolgt die Beschreibung des Systementwurfes. Der Leser muss diesem Kapitel
%die Grundstruktur des Softwaresystems entnehmen können. Soweit notwendig, sind
%auch komplexere Teilsysteme funktional zu zerlegen.
%Der Leser entnimmt hier....

\subsection{Classdiagram}
%classdiagram

\subsection{Design Decisions}

\subsubsection{}

\section{Component Design}
%Wichtig ist hier, ein einheitliches Schema zu verwenden, damit die Beschreibung konsistent ist – gerade wenn mehrere Software-Entwickler daran arbeiten.
%Da diese Dokumentation sehr änderungsanfällig ist, empfiehlt es sich ab einer gewissen Projektgröße die nötigen Informationen in den Programmcode als formatierten
%Kommentar aufzunehmen. Mit speziellen Werkzeugen (z.B. Doxygen) ist es möglich,
%die Kommentare zu extrahieren und somit ohne Mehraufwand immer eine aktuelle
%Dokumentation zu haben.



%\section{File Description}
%Hier werden die verwendeten Dateien beschrieben, sofern Informationen aus externen
%Dateien gelesen bzw. in diese geschrieben werden. Eine solche Beschreibung umfasst...

\section{Test Protocol}

Im Testprotokoll werden die Testdaten und die Testergebnisse für alle Testfälle beschrieben. Weiters muss die Testumgebung angeführt sein (welches Testframework
wurde verwendet, mit welchen Komponentenversionen wurde gestestet, welche Stubs
wurden verwendet, etc). Wenn die Komponenten und Subsysteme getrennt getestet
wurden, ist die Testprotokollierung für die Komponenten getrennt anzugeben. Weiters sind identifizierte Schwachstellen und Probleme festzuhalten.

\section{Tables and Graphs}

Ergänzende und unterstützende Tabellen und Diagramme können am Ende der Systemdokumentation angefügt werden.

% Zur Info: Tabellen am besten in Excel erstellen und dann als Bild hier einfügen - Tabellen können in Latex sehr unangenehm werden!!
% Befehl um Bilder einzufügen:
%\includegraphics{Relativer/Pfad/zum/Bild.Endung}

\section{Software Quality Metrics}

Metriken dienen dazu, die Qualität von Software zu messen. Für C++ gibt es hier
etwa das frei verfügbare Programm CCCC, welches für Linux und Windows unter
der Adresse http://sourceforge.net/projects/cccc verfügbar ist. Für einen gegebenen
Souce Code werden die Metriken ermittelt und ausgegeben. Eine (kompakte und auf
das Wesentliche gekürzte) Ausgabe kann hier auf freiwilliger Basis aufgenommen
werden und dient als Referenz bei Wartungsarbeiten (Degeneration des Codes und
Designs).
\section{Source Code}

\subsection{Class Carpool}
\subsubsection{Carpool.h}
\sourceCode{./Carpool/Carpool/Carpool.h}
\subsubsection{Carpool.cpp}
\sourceCode{./Carpool/Carpool/Carpool.cpp}

\subsection{Class Vehicle}
\subsubsection{Vehicle.h}
\sourceCode{./Carpool/Carpool/Vehicle.h}
\subsubsection{Vehicle.cpp}
\sourceCode{./Carpool/Carpool/Vehicle.cpp}

\subsection{Class Logbook}
\subsubsection{Logbook.h}
\sourceCode{./Carpool/Carpool/Logbook.h}
\subsubsection{Logbook.cpp}
\sourceCode{./Carpool/Carpool/Logbook.cpp}

\subsection{Class Motorcycle}
\subsubsection{Motorcycle.h}
\sourceCode{./Carpool/Carpool/Motorcycle.h}
\subsubsection{Motorcycle.cpp}
\sourceCode{./Carpool/Carpool/Motorcycle.cpp}

\subsection{Class Car}
\subsubsection{Car.h}
\sourceCode{./Carpool/Carpool/Car.h}
\subsubsection{Car.cpp}
\sourceCode{./Carpool/Carpool/Car.cpp}

\subsection{Class Truck}
\subsubsection{Truck.h}
\sourceCode{./Carpool/Carpool/Truck.h}
\subsubsection{Truck.cpp}
\sourceCode{./Carpool/Carpool/Truck.cpp}

% Um Quellcode einzufügen einfach diesen Befehl verwenden:
%\sourceCode{Relativer/Pfad/zum/SourceCode.Endung}

\end{document}