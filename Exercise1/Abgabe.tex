
\documentclass[12pt,a4paper]{scrartcl}

\usepackage[a4paper, left=2cm, right=1cm, bottom=1cm, top=1cm, includeheadfoot]{geometry}
\usepackage[ngerman]{babel}
\usepackage[utf8]{inputenc} % comment this if you uncomment utf8x
%\usepackage[utf8x]{inputenc} % uncomment this if there are problems with 'ä', 'ü', 'ö'
\usepackage{ucs}
\usepackage[usenames,dvipsnames]{xcolor}
\usepackage[fleqn]{amsmath}
\usepackage{amsfonts}
\usepackage{amssymb}
\usepackage{color}
\usepackage{listings}
\usepackage{hyperref}
\usepackage{amsfonts}
\usepackage{listings}
\usepackage{scrpage2}
\usepackage{graphicx}


\definecolor{mygray}{rgb}{0.9,0.9,0.9}
\lstset{language=[Visual]Basic, morekeywords={param, local}}


\lstset{
   literate={ö}{{\"o}}1
           {ä}{{\"a}}1
           {ü}{{\"u}}1
           {ß}{{\ss}}1
           {é}{{\'e}}1,
   inputencoding=ansinew,
   extendedchars=true,
   basicstyle=\scriptsize\ttfamily,
   numberstyle=\scriptsize,
   breaklines=true,
   tabsize=2,
   numbersep=5pt
}
\lstdefinestyle{customcpp}{
   language=C++,
   backgroundcolor=\color{mygray},
   numbers=left,
   keywordstyle=\color{blue}\bfseries,
   stringstyle=\color{BrickRed}\ttfamily,
   commentstyle=\color{OliveGreen}\ttfamily,
   showspaces=false,
   showstringspaces=false,
   showtabs=false
}
\lstdefinestyle{customoutput}{
   backgroundcolor=\color{mygray},
   numbers=none,
   showspaces=false,
   showtabs=false
}

\newcommand{\sourceCode}[1]{\lstinputlisting[style=customcpp]{#1}} %beinhaltet alle benötigten Packages etc.
\begin{document}

\title{SDP - Uebung 01} % Übungsname und Nummer angeben
\subtitle{Wintersemester 2019/20} % Semester angeben oder auskommentieren, falls nicht erwünscht
\author{
Viktoria Streibl - S1810306013\\
  Daniel Weyrer - S1820306047
} % Autorenname
%\date{\today}
\date{\today} % Das heutige Datum automatisch einfügen

\maketitle % Titelseite erstellen

\newpage
\tableofcontents % Inhaltsverzeichnis erstellen
\newpage

% Kure Befehlsreferenz
%\section{} erstellt eine Überschrift
%während
%\subsection{} eine Unterüberschrift erstellt
% Eine neue Seite wird mit \newpage erstellt

\section{Organisatorisches}

Hier werden organisatorische Dinge angeführt, etwa:
\begin{itemize}
	\item das Team und seine Mitarbeiter,
	\item die Aufteilung und die Verantwortungsbereiche der Mitarbeiter,
	\item der geschätzte Aufwand in Mh,
\end{itemize}

% Für eine numerierte Aufzählung verwendet man 

%\begin{enumerate}
%	\item 
%\end{enumerate}


\section{Anforderungsdefinition (Systemspezifikation)}

Hier wird die Aufgabenstellung beschrieben. Sie ist gegebenfalls aus der Systemspezifikation abzuleiten und in kompakter Form zu halten. Man sollte hier in kurzen und
prägnanten Worten lesen können, worum es überhaupt geht. Neben den Anforderungen und Zielen steht hier alles, was von fundamentaler Bedeutung für das Verständnis
eines nicht versierten Lesers (der aber sehr wohl über einschlägige Kenntnisse verfügt) nötig ist. Alles andere entfällt hier.

\section{Systementwurf}

Hier erfolgt die Beschreibung des Systementwurfes. Der Leser muss diesem Kapitel
die Grundstruktur des Softwaresystems entnehmen können. Soweit notwendig, sind
auch komplexere Teilsysteme funktional zu zerlegen.
Der Leser entnimmt hier....


\section{Komponentenentwurf}

...

Wichtig ist hier, ein einheitliches Schema zu verwenden, damit die Beschreibung konsistent ist – gerade wenn mehrere Software-Entwickler daran arbeiten.
Da diese Dokumentation sehr änderungsanfällig ist, empfiehlt es sich ab einer gewissen Projektgröße die nötigen Informationen in den Programmcode als formatierten
Kommentar aufzunehmen. Mit speziellen Werkzeugen (z.B. Doxygen) ist es möglich,
die Kommentare zu extrahieren und somit ohne Mehraufwand immer eine aktuelle
Dokumentation zu haben.


\section{Dateibeschreibung}

Hier werden die verwendeten Dateien beschrieben, sofern Informationen aus externen
Dateien gelesen bzw. in diese geschrieben werden. Eine solche Beschreibung umfasst...

\section{Testprotokollierung}

Im Testprotokoll werden die Testdaten und die Testergebnisse für alle Testfälle beschrieben. Weiters muss die Testumgebung angeführt sein (welches Testframework
wurde verwendet, mit welchen Komponentenversionen wurde gestestet, welche Stubs
wurden verwendet, etc). Wenn die Komponenten und Subsysteme getrennt getestet
wurden, ist die Testprotokollierung für die Komponenten getrennt anzugeben. Weiters sind identifizierte Schwachstellen und Probleme festzuhalten.

\section{Tabellen und Diagramme}

Ergänzende und unterstützende Tabellen und Diagramme können am Ende der Systemdokumentation angefügt werden.

% Zur Info: Tabellen am besten in Excel erstellen und dann als Bild hier einfügen - Tabellen können in Latex sehr unangenehm werden!!
% Befehl um Bilder einzufügen:
%\includegraphics{Relativer/Pfad/zum/Bild.Endung}

\section{Software-Qualitätsmetriken}

Metriken dienen dazu, die Qualität von Software zu messen. Für C++ gibt es hier
etwa das frei verfügbare Programm CCCC, welches für Linux und Windows unter
der Adresse http://sourceforge.net/projects/cccc verfügbar ist. Für einen gegebenen
Souce Code werden die Metriken ermittelt und ausgegeben. Eine (kompakte und auf
das Wesentliche gekürzte) Ausgabe kann hier auf freiwilliger Basis aufgenommen
werden und dient als Referenz bei Wartungsarbeiten (Degeneration des Codes und
Designs).

\section{Quellcode}

\subsection{Erster Header}

% Um Quellcode einzufügen einfach diesen Befehl verwenden:
%\sourceCode{Relativer/Pfad/zum/SourceCode.Endung}

\end{document}