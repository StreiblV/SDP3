\include{Definition} %beinhaltet alle benötigten Packages etc.
\begin{document}
\graphicspath{{./}}

\includepdf[pages=-]{../Angabe.pdf}

\title{SDP - Exercise 08} % Übungsname und Nummer angeben
\subtitle{winter semester 2019/20} % Semester angeben oder auskommentieren, falls nicht erwünscht
\author{
Viktoria Streibl - S1810306013\\
  Daniel Weyrer - S1820306044
} % Autorenname
\date{\today} % Das heutige Datum automatisch einfügen

\maketitle % Titelseite erstellen

\newpage
\tableofcontents % Inhaltsverzeichnis erstellen
\newpage

\ihead{Viktoria Streibl}
\ohead{Daniel Weyrer}
\chead{SDP3-UE Uebung 08}

\section{Organizational}
\subsection{Team}
\begin{itemize}
	\item Viktoria 	Streibl 		- 	S1810306013
	\item Daniel 	Weyrer		-	S1820306044
\end{itemize}

\subsection{Roles and responsibilities}
\subsubsection{Jointly}
\begin{itemize}
	\item Planning
	\item Documentation
	\item Systemdocumentation
	\item Class Diagram
\end{itemize}

\subsubsection{Viktoria Streibl}
\begin{itemize}
	\item Client
	\item Hexapod
	\item IRobot
	\item Object
	\item Robot
	\item Wheelbot
\end{itemize}

\subsubsection{Daniel Weyrer}
\begin{itemize}
	\item Control
	\item Forward
	\item ICommand
	\item MacroMovement
	\item TestDriver
	\item TurnLeft
	\item TurnRight
\end{itemize}

\subsection{Effort}

\subsubsection {Viktoria Streibl}
\begin{itemize}
	\item estimated: 6 ph 
	\item actually: 3 ph
\end{itemize}

\subsubsection {Daniel Weyrer}
\begin{itemize}
	\item estimated: 6 ph 
	\item actually: 6 ph
\end{itemize}

\section{Requirenment Definition(System Specification)}
This program should reflect robot control. Different robots should receive different commands and react accordingly.
The following commands were requested:
\begin{itemize}
	\item Turn Left
	\item Turn Right
	\item Forward
	\item MacroMovement - get multiple commands
\end{itemize}

\section{System Design}
\subsection{Design Decisions}
\subsubsection{Case-Statements in TurnLeft and TurnRight}
To make the commands independent from each other (so there`s no need to implement a TurnRight when there is a TurnLeft and the other way round), both Commands have the complete Implementation for their Undo-Command!

\newpage
\subsection{Classdiagram}
\includegraphics[scale=0.35, angle=90]{../ClassDiagram.pdf}

\section{Component Design}
\subsection{Client}
The client class has only one function which tests the robot control system. It creates two different robots, a hexapod and a wheelbot and sends them commands.
Then calls the robots and check there position and data.

\subsection{Control}
The modul control has to lists. The command-list, it collects all commands and the the undo-list to undo all already received commands.
It has also following functions:
\begin{itemize}
	\item Add Comment
	\subitem This function gets a command and add it to the command list
	\item Start
	\subitem This functions execute all of the commands ins the commandlist
	\item Undo
	\subitem This funtion adds a command to the undo-list
	\item Reset
	\subitem This function removes all added commands from the lists
\end{itemize}

\subsection{Hexapod}
This modul is a robot type and inherit  from the class Robot. It has an constructor which has the name of the robot as parameter and set it.
It also has a function Info, which prints all the informations of this robot.

\subsection{Wheelbot}
This modul is a robot type and inherit  from the class Robot. It has an constructor which has the name of the robot as parameter and set it.
It also has a function Info, which prints all the informations of this robot.

\subsection{ICommand}
This is an Interface and stores the robot, which is given via the constructor.
It declerates the functions Execute and Unexecute.

\subsection{IRobot}
IRobot is an interface. It declerates the method Info with and ostream parameter.

\subsection{Robot}
This modul has an enum class DIR, it contains the directions north, south, east and west.
The class Robot inherite from the Interface IRobot. It stores the name, the x-position, the y-position and the direction.
It also has following methods:
\begin{itemize}
	\item Set/Get-Direction
	\subitem The setter gets a direction and saves it. The getter returns the stored direction of the robot.
	\item Set/Get-PosX
	\subitem The setter gets the x-position and saves it. The getter returns the stored direction of the robot.
	\item Set/Get-PosY
	\subitem The setter gets the y-position and saves it. The getter returns the stored direction of the robot.
	\item Set/Get-Name
	\subitem The setter gets the name of the robot and saves it. The getter returns the stored direction of the robot.
\end{itemize}

\subsection{Forward}
This class inherite from ICommand.
It overrides the functions Execute and Unexecute.  These functions increase or decrease the position x,y depending on the direction.

\subsection{TurnLeft}
This class inherite from ICommand.
It overrides the functions Execute and Unexecute.  These functions change the direction by 1 clockwise.

\subsection{TurnRight}
This class inherite from ICommand.
It overrides the functions Execute and Unexecute.  These functions change the direction by 1 counterclockwise.

\subsection{MacroMovement}
This class inherite from ICommand.
It overrides the functions Execute and Unexecute. 
The Execute function execute all stored commands.
The Unexecute function undo the commands.

\section{Source Code}

\subsection{Client}
\subsubsection{Client.h}
\sourceCode{../../Robot/Robot/Client.h}
\subsubsection{Client.cpp} 
\sourceCode{../../Robot/Robot/Client.cpp}
\newpage

\subsection{Control}
\subsubsection{Control.h}
\sourceCode{../../Robot/Robot/Control.h}
\subsubsection{Control.cpp}
\sourceCode{../../Robot/Robot/Control.cpp}
\newpage

\subsection{Hexapod}
\subsubsection{Hexapod.h}
\sourceCode{../../Robot/Robot/Hexapod.h}
\subsubsection{Hexapod.cpp}
\sourceCode{../../Robot/Robot/Hexapod.cpp}
\newpage

\subsection{Wheelbot}
\subsubsection{Wheelbot.h}
\sourceCode{../../Robot/Robot/Wheelbot.h}
\subsubsection{Wheelbot.cpp}
\sourceCode{../../Robot/Robot/Wheelbot.cpp}
\newpage

\subsection{ICommand}
\subsubsection{ICommand.h}
\sourceCode{../../Robot/Robot/ICommand.h}
\subsubsection{ICommand.cpp}
\sourceCode{../../Robot/Robot/ICommand.cpp}
\newpage

\subsection{IRobot}
\subsubsection{IRobot.h}
\sourceCode{../../Robot/Robot/IRobot.h}
\subsubsection{IRobot.cpp}
\sourceCode{../../Robot/Robot/IRobot.cpp}
\newpage

\subsection{Robot}
\subsubsection{Robot.h}
\sourceCode{../../Robot/Robot/Robot.h}
\subsubsection{Robot.cpp}
\sourceCode{../../Robot/Robot/Robot.cpp}
\newpage

\subsection{Forward}
\subsubsection{Forward.h}
\sourceCode{../../Robot/Robot/Forward.h}
\subsubsection{Forward.cpp}
\sourceCode{../../Robot/Robot/Forward.cpp}
\newpage

\subsection{TurnLeft}
\subsubsection{TurnLeft.h}
\sourceCode{../../Robot/Robot/TurnLeft.h}
\subsubsection{TurnLeft.cpp}
\sourceCode{../../Robot/Robot/TurnLeft.cpp}
\newpage

\subsection{TurnRight}
\subsubsection{TurnRight.h}
\sourceCode{../../Robot/Robot/TurnRight.h}
\subsubsection{TurnRight.cpp}
\sourceCode{../../Robot/Robot/TurnRight.cpp}
\newpage

\subsection{MacroMovement}
\subsubsection{MacroMovement.h}
\sourceCode{../../Robot/Robot/MacroMovement.h}
\subsubsection{MacroMovement.cpp}
\sourceCode{../../Robot/Robot/MacroMovement.cpp}
\newpage

\subsection{Object}
\subsubsection{Object.h}
\sourceCode{../../Robot/Robot/Object.h}
\newpage

\subsection{TestDriver}
\subsubsection{TestDriver.cpp}
\sourceCode{../../Robot/Robot/TestDriver.cpp}
\newpage


\end{document}