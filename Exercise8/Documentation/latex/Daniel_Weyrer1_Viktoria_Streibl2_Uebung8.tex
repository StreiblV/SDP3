\include{Definition} %beinhaltet alle benötigten Packages etc.
\begin{document}
\graphicspath{{./}}

\includepdf[pages=-]{../Angabe.pdf}

\title{SDP - Exercise 08} % Übungsname und Nummer angeben
\subtitle{winter semester 2019/20} % Semester angeben oder auskommentieren, falls nicht erwünscht
\author{
Viktoria Streibl - S1810306013\\
  Daniel Weyrer - S1820306044
} % Autorenname
\date{\today} % Das heutige Datum automatisch einfügen

\maketitle % Titelseite erstellen

\newpage
\tableofcontents % Inhaltsverzeichnis erstellen
\newpage

\ihead{Viktoria Streibl}
\ohead{Daniel Weyrer}
\chead{SDP3-UE Uebung 08}

\section{Organizational}
\subsection{Team}
\begin{itemize}
	\item Viktoria 	Streibl 		- 	S1810306013
	\item Daniel 	Weyrer		-	S1820306044
\end{itemize}

\subsection{Roles and responsibilities}
\subsubsection{Jointly}
\begin{itemize}
	\item Planning
	\item Documentation
	\item Systemdocumentation
	\item Class Diagram
\end{itemize}

\subsubsection{Viktoria Streibl}
\begin{itemize}
	\item Client
	\item Hexapod
	\item IRobot
	\item Object
	\item Robot
	\item Wheelbot
\end{itemize}

\subsubsection{Daniel Weyrer}
\begin{itemize}
	\item Control
	\item Forward
	\item ICommand
	\item MacroMovement
	\item TestDriver
	\item TurnLeft
	\item TurnRight
\end{itemize}

\subsection{Effort}

\subsubsection {Viktoria Streibl}
\begin{itemize}
	\item estimated: 6 ph 
	\item actually: 3 ph
\end{itemize}

\subsubsection {Daniel Weyrer}
\begin{itemize}
	\item estimated: 6 ph 
	\item actually: 6 ph
\end{itemize}

\section{Requirenment Definition(System Specification)}
This program should reflect robot control. Different robots should receive different commands and react accordingly.
The following commands were requested:
\begin{itemize}
	\item Turn Left
	\item Turn Right
	\item Forward
	\item MacroMovement - get multiple commands
\end{itemize}

\section{System Design}
\newpage
\subsection{Classdiagram}
\includegraphics[scale=0.35, angle=90]{../ClassDiagram.pdf}

\subsection{Design Decisions}
\subsubsection{Case-Statements in TurnLeft and TurnRight}
To make the commands independent from each other (so there`s no need to implement a TurnRight when there is a TurnLeft and the other way round), both Commands have the complete Implementation for their Undo-Command!

\section{Component Design}
\subsection{Client}

\subsection{Control}

\subsection{Hexapod}

\subsecction{Wheelbot}

\subsection{ICommand}

\subsection{IRobot}

\subsection{Robot}

\subsection{Forward}

\subsection{TurnLeft}

\subsection{TurnRight}

\subsection{MacroMovement}


\section{Source Code}
\subsection{Client}
\subsubsection{Client.h}
\sourceCode{../../Robot/Robot/Client.h}
\subsubsection{Client.cpp} 
\sourceCode{../../Robot/Robot/Client.cpp}
\newpage

\subsection{Control}
\subsubsection{Control.h}
\sourceCode{../../Robot/Robot/Control.h}
\subsubsection{Control.cpp}
\sourceCode{../../Robot/Robot/Control.cpp}
\newpage

\subsection{Hexapod}
\subsubsection{Hexapod.h}
\sourceCode{../../Robot/Robot/Hexapod.h}
\subsubsection{Hexapod.cpp}
\sourceCode{../../Robot/Robot/Hexapod.cpp}
\newpage

\subsection{Wheelbot}
\subsubsection{Wheelbot.h}
\sourceCode{../../Robot/Robot/Wheelbot.h}
\subsubsection{Wheelbot.cpp}
\sourceCode{../../Robot/Robot/Wheelbot.cpp}
\newpage

\subsection{ICommand}
\subsubsection{ICommand.h}
\sourceCode{../../Robot/Robot/ICommand.h}
\subsubsection{ICommand.cpp}
\sourceCode{../../Robot/Robot/ICommand.cpp}
\newpage

\subsection{IRobot}
\subsubsection{IRobot.h}
\sourceCode{../../Robot/Robot/IRobot.h}
\subsubsection{IRobot.cpp}
\sourceCode{../../Robot/Robot/IRobot.cpp}
\newpage

\subsection{Robot}
\subsubsection{Robot.h}
\sourceCode{../../Robot/Robot/Robot.h}
\subsubsection{Robot.cpp}
\sourceCode{../../Robot/Robot/Robot.cpp}
\newpage

\subsection{Forward}
\subsubsection{Forward.h}
\sourceCode{../../Robot/Robot/Forward.h}
\subsubsection{Forward.cpp}
\sourceCode{../../Robot/Robot/Forward.cpp}
\newpage

\subsection{TurnLeft}
\subsubsection{TurnLeft.h}
\sourceCode{../../Robot/Robot/TurnLeft.h}
\subsubsection{TurnLeft.cpp}
\sourceCode{../../Robot/Robot/TurnLeft.cpp}
\newpage

\subsection{TurnRight}
\subsubsection{TurnRight.h}
\sourceCode{../../Robot/Robot/TurnRight.h}
\subsubsection{TurnRight.cpp}
\sourceCode{../../Robot/Robot/TurnRight.cpp}
\newpage

\subsection{MacroMovement}
\subsubsection{MacroMovement.h}
\sourceCode{../../Robot/Robot/MacroMovement.h}
\subsubsection{MacroMovement.cpp}
\sourceCode{../../Robot/Robot/MacroMovement.cpp}
\newpage

\subsection{Object}
\subsubsection{Object.h}
\sourceCode{../../Robot/Robot/Object.h}
\newpage

\subsection{TestDriver}
\subsubsection{TestDriver.cpp}
\sourceCode{../../Robot/Robot/TestDriver.cpp}
\newpage


\end{document}